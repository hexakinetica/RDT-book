% ===================================================
%  ДОКУМЕНТ И ЯЗЫК
% ===================================================
\documentclass[11pt, b5paper, oneside]{book}

\usepackage{polyglossia}
\setdefaultlanguage{english}
\setotherlanguage{russian}

% ===================================================
%  ШРИФТЫ (СОВРЕМЕННЫЙ ПОДХОД ДЛЯ LUALATEX)
% ===================================================
\usepackage{fontspec}
\usepackage{unicode-math}

\setmainfont{Libertinus Serif}
\setsansfont{IBM Plex Sans}
\setmonofont{IBM Plex Mono}
\setmathfont{Latin Modern Math}

\usepackage{textcomp}
\usepackage{alltt}
% ===================================================
%  ГЕОМЕТРИЯ СТРАНИЦЫ И ТИПОГРАФИКА
% ===================================================
\usepackage[b5paper,margin=18mm]{geometry}
\usepackage{setspace}
\setstretch{1.05}
\usepackage{microtype} % Улучшает внешний вид текста
\usepackage{ragged2e}  % Для \RaggedRight, \Centering, \RaggedLeft

% ===================================================
%  КОЛОНТИТУЛЫ
% ===================================================
\usepackage{fancyhdr}
\pagestyle{fancy}
\fancyhf{}
\fancyhead[LE,RO]{\thepage}
\fancyhead[RE]{\leftmark}
\fancyhead[LO]{\rightmark}
\renewcommand{\headrulewidth}{0.4pt}
\renewcommand{\footrulewidth}{0pt}

% ===================================================
%  ОФОРМЛЕНИЕ ЗАГОЛОВКОВ И СПИСКОВ
% ===================================================
\usepackage{titlesec}
\titleformat{\chapter}[display]
  {\normalfont\huge\bfseries}
  {\chaptername~\thechapter}{20pt}{\Huge}

\usepackage{enumitem}
\setlist[itemize]{nosep,topsep=4pt,parsep=2pt,partopsep=0pt,leftmargin=2em}

% ===================================================
%  ЦВЕТА, ГРАФИКА И ИКОНКИ
% ===================================================
\usepackage{xcolor}
\usepackage{graphicx}
\usepackage{tikz-cd}
\usepackage{fontawesome5}
\usepackage{scalerel}
\usepackage{xparse}

% --- Кастомные иконки ---
\NewDocumentCommand\warning{}{\textcolor{orange}{\faIcon{exclamation-triangle}}}
\NewDocumentCommand\information{}{\textcolor{blue}{\faIcon{info-circle}}}
\NewDocumentCommand\greenCheck{}{\textcolor{green!60!black}{\faIcon{check-circle}}}

% ===================================================
%  ПЛАВАЮЩИЕ ОБЪЕКТЫ И ПОДПИСИ
% ===================================================
\usepackage{caption}
\usepackage{float}      % Для управления плавающими окружениями (например, [H])
\usepackage{placeins}   % Для команды \FloatBarrier
\captionsetup{justification=justified, singlelinecheck=true}

% ===================================================
%  ОФОРМЛЕНИЕ КОДА (MINTED + LISTINGS)
% ===================================================
% Пакет minted для основной подсветки
\usepackage{minted}

% Пакет listings для кастомной команды \hcode
\usepackage{listings} 

% --- Общие настройки для minted ---
\definecolor{mintedbg}{HTML}{FAFAFA}
\setminted{
  bgcolor=mintedbg,
  frame=lines,
  framerule=0.4pt,
  framesep=3mm,
  fontsize=\footnotesize,
  breaklines=true,
  linenos=true,
  numbersep=5pt,
  style=default
}

% --- Создание плавающего окружения для листингов кода minted ---
\newfloat{listing}{htbp}{lol}
\floatname{listing}{Listing} % Или "Листинг"
\captionsetup[listing]{
    skip=-10pt,
    belowskip=10pt
}

% --- Стиль и команда для инлайн-кода (\hcode) через listings ---
\definecolor{hcode_bg_color}{HTML}{FEFDC4} % Чуть менее яркий желтый
\lstdefinestyle{highlighted_inline_force_break}{
  backgroundcolor=\color{hcode_bg_color},
  basicstyle=\ttfamily\color{blue}\footnotesize,
  breaklines=true,
  breakatwhitespace=false,
  postbreak=\mbox{\textcolor{red}{$\hookrightarrow$}\space},
}
\newcommand{\hcode}[1]{\lstinline[style=highlighted_inline_force_break]|#1|}

% ===================================================
%  ОФОРМЛЕНИЕ ТАБЛИЦ (TABULARRAY)
% ===================================================
\usepackage{tabularray}
\UseTblrLibrary{booktabs} % Подключает функционал booktabs, \usepackage{booktabs} не нужен
\usepackage{etoolbox}

\newcommand{\tabitem}{\textbullet\hspace{0.5em}}
\AtBeginEnvironment{longtable}{\small} % Уменьшает шрифт для длинных таблиц

% ===================================================
%  ЦВЕТНЫЕ БЛОКИ (TCOLORBOX)
% ===================================================
\usepackage[most]{tcolorbox}
\tcbuselibrary{skins, breakable}

% --- Цветовая палитра ---
\definecolor{NavGreen}{RGB}{46, 139, 87}
\definecolor{NavGreenBack}{RGB}{240, 248, 240}

%\definecolor{DangerRed}{RGB}{220, 53, 69}
%\definecolor{DangerRedBack}{RGB}{253, 237, 237}

%
\definecolor{DangerRed}{RGB}{190, 0, 0}    % Новый глубокий красный для рамки
\definecolor{DangerRedBack}{RGB}{255, 200, 200} % Новый светло-красный для фона

% Переопределение цветов для DangerBox на "ядовито красный, но светлее"
%\definecolor{DangerRed}{RGB}{255, 69, 0}    % Яркий, "ядовитый" красно-оранжевый для рамки
\definecolor{DangerRedBack}{RGB}{255, 230, 230} % Очень светлый, мягкий розово-красный для фона

\definecolor{PrincipleBlue}{RGB}{23, 118, 207}
\definecolor{PrincipleBlueBack}{RGB}{232, 244, 253}

%\definecolor{TipOrange}{RGB}{253, 126, 20}
%\definecolor{TipOrangeBack}{RGB}{255, 249, 232}

% Переопределение цветов для TipBox на фиолетовый
\definecolor{TipOrange}{RGB}{140, 100, 180}    % Новый темно-фиолетовый для рамки (используется как 'TipPurpleFrame')
\definecolor{TipOrangeBack}{RGB}{245, 240, 255} % Новый светло-фиолетовый для фона (используется как 'TipPurpleBack')

\definecolor{InfoGray}{RGB}{108, 117, 125}
\definecolor{InfoGrayBack}{RGB}{248, 249, 250}

% --- Базовый стиль для "современных" блоков ---
\tcbset{
  bookbox/.style={
    enhanced, breakable, skin=enhanced, arc=1mm, outer arc=1mm,
    borderline west={2.5pt}{0pt}{#1},
    boxrule=0.5pt, colframe=black!15!white,
    fonttitle=\bfseries, coltitle=black!85!white,
    fontupper=\small,
    left=15pt, right=10pt, top=10pt, bottom=10pt,
    drop fuzzy shadow={black!30!white}
  }
}

% --- Кастомные "современные" блоки ---
\newtcolorbox{navigationbox}[1]{bookbox=NavGreen, colback=NavGreenBack, title={#1}}
\newtcolorbox{dangerbox}[1]{bookbox=DangerRed, colback=DangerRedBack, title={#1}}
\newtcolorbox{principlebox}[1]{bookbox=PrincipleBlue, colback=PrincipleBlueBack, title={#1}}
\newtcolorbox{tipbox}[1]{bookbox=TipOrange, colback=TipOrangeBack, title={#1}}
\newtcolorbox{infobox}[1]{bookbox=InfoGray, colback=InfoGrayBack, title={#1}}

% --- Кастомные "legacy" блоки ---
\definecolor{lightbg}{HTML}{F8F8F8}
\definecolor{lightframe}{HTML}{CCCCCC}
% \tcbset{ % <-- ВНИМАНИЕ: этот блок глобально переопределит стили выше.
%   colback=lightbg, colframe=lightframe, sharp corners, boxrule=0.4pt,
%   left=6pt, right=6pt, top=6pt, bottom=6pt, enhanced,
% }
\newtcolorbox{box_important}{colback=red!5!white, colframe=red!75!black, fonttitle=\bfseries, title=Important}
\newtcolorbox{box_tip}{colback=blue!5!white, colframe=blue!75!black, fonttitle=\bfseries, title=Advise}
\newtcolorbox{box_danger}{colback=yellow!10!white, colframe=orange!85!black, fonttitle=\bfseries, title=Danger}

% ===================================================
%  ГИПЕРССЫЛКИ, БИБЛИОГРАФИЯ И ГЛОССАРИЙ
% ===================================================
% --- Глоссарий ---
\usepackage{glossaries}
\makenoidxglossaries
\newglossaryentry{production-ready}{name={production-ready}, description={test}}
\newcommand{\term}[1]{%
    \ifglsentryexists{#1}
    {\textit{\gls{#1}}}%
    {\textbf{\textcolor{purple}{#1}}}%
}

% --- Библиография (biblatex) ---
\usepackage[backend=biber,style=ieee,sorting=none]{biblatex}
\addbibresource{bibliography.bib}

% --- Гиперссылки (hyperref) - грузить одним из последних ---
\usepackage{hyperref}
\hypersetup{
  colorlinks=true,
  linkcolor=blue,
  citecolor=magenta,
  urlcolor=blue,
  breaklinks=true % Полезно для длинных URL в библиографии
}


% Пакеты
\usepackage{fancyhdr}
\hypersetup{colorlinks=true, urlcolor=blue} % делаем ссылки синими
\usepackage{lipsum} % для текста-рыбы

%-----------------------------------------------------
% 1. ОПРЕДЕЛЕНИЕ СТИЛЕЙ КОЛОНТИТУЛОВ
%-----------------------------------------------------

% Стиль для основных глав
\fancypagestyle{main}{%
  %\fancyfoot[RE]{\thepage}
  \renewcommand{\footrulewidth}{0pt}
}

% Стиль для вводных страниц и приложений с GitHub ссылками
\fancypagestyle{github}{%
  \fancyfoot[C]{\footnotesize \texttt{github.com/hexakinetica/rdt-book \ \ \ \ \ \ \ github.com/hexakinetica/rdt-core}}
  %\fancyfoot[R]{\thepage}
  \renewcommand{\footrulewidth}{0.4pt}
}
