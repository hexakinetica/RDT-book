\newpage
\chapter*{A Note on this Draft Version}
\addcontentsline{toc}{chapter}{Note on this Draft}

\paragraph{Welcome!}

You are reading a living document. This book, like the RDT software it describes, is an open-source project under active development. The version you hold in your hands (or see on your screen) is a complete draft: the core structure, architectural concepts, and code examples are in place. However, it is not yet the final, polished product.

We chose to release it in this state deliberately, following the open-source philosophy of "release early, release often." We believe that community feedback is the most valuable tool for creating a truly great resource.

We are aware of several shortcomings in this current version. This is where you come in. If you find these or any other issues, we would be incredibly grateful if you would report them or even help us fix them.

\vspace{2ex}
\subsection*{Known Issues \& Areas for Contribution}

\begin{itemize}
    \item \textbf{Typos and Grammatical Errors:} As English is not the primary language for all contributors, you will undoubtedly find typos, awkward phrasing, and grammatical mistakes. Every corrected sentence makes the book better for the next reader.

    \item \textbf{Missing Diagrams and Illustrations:} Many sections refer to diagrams that are not yet drawn (often marked with placeholders like \texttt{Figure ??}). Visualizing complex concepts is crucial, and we welcome help in creating clear and informative illustrations.

    \item \textbf{Content Duplication:} In our effort to explain concepts thoroughly, we have sometimes repeated ourselves. You may notice similar explanations of ideas like "RT/NRT separation" or "prototype vs. product" in different chapters. We need help streamlining these sections to be more concise.

    \item \textbf{No Glossary:} A comprehensive glossary of terms (e.g., "SSOT", "HAL", "IK", "Determinism") is planned but not yet implemented. This is a critical feature for a book full of specialized terminology.

    \item \textbf{Inconsistent Terminology:} We may have used different terms for the same concept in different places (e.g., "State Bus", "Blackboard", "SSOT Object"). We need to standardize our vocabulary throughout the book.

    \item \textbf{Placeholder Code and Comments:} Some code listings might contain placeholders (`// ... to be implemented`) or comments in languages other than English. These need to be finalized and translated.
    
    \item \textbf{Lack of a "Further Reading" Section:} Each chapter would benefit from a curated list of links to articles, key papers, or other books for readers who want to dive deeper into a specific topic. This is a perfect area for community contributions.
\end{itemize}

\vspace{2ex}
\subsection*{How to Contribute}

The best way to contribute is through our GitHub repositories. Your feedback is the most valuable asset we have.

\begin{itemize}
    \item \textbf{For the book's content (text, diagrams, structure):} Please open an Issue or a Pull Request in the book repository: \\
    \texttt{github.com/hexakinetica/rdt-book}

    \item \textbf{For the RDT source code:} If you find a bug or have an idea for improving the RDT controller itself, please use the code repository: \\
    \texttt{github.com/hexakinetica/rdt-core}
\end{itemize}

Thank you for joining us on this journey. Let's build something great together.