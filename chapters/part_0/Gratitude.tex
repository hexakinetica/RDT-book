\newpage

% --- Новый, более личный заголовок ---
\chapter*{A Note}
\addcontentsline{toc}{chapter}{A Note from the Author}

\textbf{Why This Book (and Project) Exists}

This book was born from a simple observation: there is a gap between what is taught at universities and what engineers face in the real world of industrial robotics. We have walked that path ourself — from the excitement of seeing a ROS script finally work, to the cold sweat of debugging a production-line failure at 3 a.m.

We’ve seen dozens of projects that looked flawless on a demo stand but collapsed the moment they met real-world constraints. We’ve seen elegant abstractions turn into unmaintainable monsters, and simple but robust solutions run reliably for years. Our motivation was to systematize this experience. We didn’t want to write yet another kinematics textbook. We wanted to create a kind of “field manual” for engineers.

More importantly, We wanted to build something real. That’s why the \hcode{Robot Development Toolkit (RDT)} project was started. This book is the story of that project — a guide to its architecture, its code, and the lessons learned along the way. It is built on a core belief: theory is essential, but working code is the ultimate truth.

This is an open-source endeavor. If this book and the RDT code help even one engineer avoid the mistakes we made and build a more reliable, elegant system, I will consider our goal achieved.

\begin{flushright}
    \textit{— Aleksandr L.}
\end{flushright}

\begin{center}
    ***
\end{center}
\vspace{1ex}

\textbf{Acknowledgments}

This project is the result of a long and challenging marathon, and it would not have been possible without the support of many people.

I am grateful to my colleagues and mentors, who over the years have shared their invaluable experience in industrial automation. Your lessons, debates, and real-world engineering challenges form the basis of many principles discussed in this book.

Finally, a huge thank you to the entire open-source community. To the developers of incredible tools like Eigen, KDL, and Qt. And to the thousands of engineers who share their knowledge on Stack Overflow, GitHub, and personal blogs. The RDT project stands on the shoulders of these giants.