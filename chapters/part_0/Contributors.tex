\newpage
\chapter*{The RDT Community} % Более "опенсорсное" название
\addcontentsline{toc}{chapter}{Project Contributors}

% Вводный абзац с прямым призывом и двумя ссылками
The Robot Development Toolkit (RDT) is a community-driven, open-source project. This book and the controller it describes are the result of collaborative efforts by an international community of engineers, developers, and robotics enthusiasts. 

We believe that great software is built together. We invite you to become a part of our community. Your ideas, bug reports, fixes, and code contributions make project thrive.

\begin{center}
    \large
    \textbf{Join the Conversation and Contribute:}\\[1ex]
    Book \& Documentation: \href{https://github.com/hexakinetica/rdt-book}{\texttt{github.com/hexakinetica/rdt-book}}\\[1ex]
    Controller Source Code: \href{https://github.com/hexakinetica/rdt-code}{\texttt{github.com/hexakinetica/rdt-code}}
\end{center}

Below are just a few of the key contributors whose work has had a significant impact. A full list can always be found in the project's commit history.

% --- Core Team ---
\subsection*{Core Team}

% Используем тот же стиль, но с небольшими улучшениями форматирования
\begin{description}
    \item[Ivan (\href{https://github.com/ivanov-ivan-rdt}{@Ivanov})] \hfill \textbf{Core Architecture, RT Domain} \\
    \textit{Designed the foundation of the multithreaded architecture and real-time kernel; implemented core synchronization mechanisms.}

    \item[Peter (\href{https://github.com/petrov-petr-rdt}{@Petrov})] \hfill \textbf{Kinematics and Planning} \\
    \textit{Implemented the mathematical core, including adapters for KDL and Eigen, and developed interpolation and velocity profiling algorithms.}

    \item[Maria (\href{https://github.com/sidorova-maria-rdt}{@Sidorova})] \hfill \textbf{GUI and UX} \\
    \textit{Designed and developed the entire Qt-based graphical interface, including the 3D visualizer and control panels, making the system intuitive and user-friendly.}
\end{description}

% --- Contributors List ---
\subsection*{Key Contributors} % "Key Contributors" звучит более весомо

We are also immensely grateful to the following contributors for their specific bug fixes, feature suggestions, and valuable feedback:

% Используем формат, похожий на Core Team для единообразия
\begin{itemize}
    \item \textbf{Sergey Kuznetsov (\href{https://github.com/kuznetsov-sergey-rdt}{@kuznetsov-sergey})}: For developing a multi-layered testing strategy and integrating CI/CD workflows.
    \item ... and many others whose contributions can be found in the project's commit history.
\end{itemize}